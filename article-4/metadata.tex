% DO NOT EDIT - automatically generated from metadata.yaml

\def \codeURL{https://github.com/dido1998/h-detach/tree/v1.0}
\def \codeDOI{10.5281/zenodo.2657361}
\def \dataURL{}
\def \dataDOI{}
\def \editorNAME{Koustuv Sinha}
\def \editorORCID{0000-0002-2803-9236}
\def \reviewerINAME{Anonymous reviewers}
\def \reviewerIORCID{}
\def \reviewerIINAME{}
\def \reviewerIIORCID{}
\def \dateRECEIVED{04 May 2019}
\def \dateACCEPTED{04 May 2019}
\def \datePUBLISHED{22 May 2019}
\def \articleTITLE{[Re] h-detach: Modifying the LSTM gradient towards better optimization}
\def \articleTYPE{Replication}
\def \articleDOMAIN{Machine Learning}
\def \articleBIBLIOGRAPHY{bibliography.bib}
\def \articleYEAR{2019}
\def \reviewURL{https://github.com/reproducibility-challenge/iclr_2019/pull/148}
\def \articleABSTRACT{Recurrent neural networks have been widely used for processing sequences. Recurrent neural networks     are known for their exploding and vanishing gradient problem (EVGP). This problem becomes more evident  in tasks where the information needed to correctly solve them exist over long time scales, because EVGP prevents important gradient components from being back-propagated adequately over a large number of steps. This paper aims to reproduce the results of the paper \cite{kanuparthi2018hdetach}.\cite{kanuparthi2018hdetach} introduces a stochastic algorithm (h-detach) to mitigate the EVGP problem in Long Short Term Memory networks. Long Short Term Memory networks – usually just called ``LSTMs"" – are a special kind of RNN, capable of learning long-term dependencies.}
\def \replicationCITE{}
\def \replicationBIB{kanuparthi2018hdetach}
\def \replicationURL{https://arxiv.org/pdf/1810.03023.pdf}
\def \replicationDOI{}
\def \contactNAME{Aniket Didolkar}
\def \contactEMAIL{}
\def \articleKEYWORDS{rescience c, rescience x, Machine Learning, ICLR Reproducibility Challenge, ICLR 2019}
\def \journalNAME{ReScience C}
\def \journalVOLUME{5}
\def \journalISSUE{2}
\def \articleNUMBER{4}
\def \articleDOI{}
\def \authorsFULL{Aniket Didolkar}
\def \authorsABBRV{A. Didolkar}
\def \authorsSHORT{Didolkar}
\title{\articleTITLE}
\date{}
\author[1,\orcid{0000-0001-9183-3144}]{Aniket Didolkar}
\affil[1]{Manipal Academy of Higher Education, Manipal, India}
